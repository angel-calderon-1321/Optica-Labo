\documentclass[11pt,a4paper]{article}
\usepackage[latin1]{inputenc}
\usepackage[T1]{fontenc}
\usepackage[spanish]{babel}
\usepackage{amsmath}
\usepackage{amsfonts}
\usepackage{amssymb}
\usepackage{graphicx}
\usepackage[colorlinks=true, linkcolor=red, citecolor=red, urlcolor=blue]{hyperref}
\usepackage[total={16cm,24cm},centering]{geometry}
\parskip= 5mm %espacio entre p�rrafos
\usepackage{caption} %caption personalizado
\date{}
\author{Goodman}
\title{Difracci�n de Frenel y Fraunhofer}
\begin{document}
\maketitle
\newpage
\tableofcontents
\newpage

En el cap�tulo anterior se presentaron los resultados de la teor�a de la difracci�n escalar en sus formas m�s generales. Ahora se centra la atenci�n en ciertas aproximaciones a la teor�a general, aproximaciones que permitir�n que los c�lculos de los patrones de difracci�n se reduzcan a manipulaciones matem�ticas comparativamente simples. Estas aproximaciones, que se realizan com�nmente en muchos campos que tratan la propagaci�n de ondas, se denominar�n aproximaciones de Fresnel y Fraunhofer. De acuerdo con nuestra visi�n del fen�meno de propagaci�n de ondas como un "sistema", intentaremos encontrar aproximaciones que sean v�lidas para una amplia clase de distribuciones de campos de entrada.

\section{BACKGROUND}

En esta secci�n preparamos al lector para los c�lculos que se realizar�n a continuaci�n. Se introduce el concepto de intensidad de un campo de ondas y se presenta el principio de Huygens-Fresnel, del que se derivan las aproximaciones, de una forma especialmente adecuada para la aproximaci�n.

\subsection{Intensidad de campo de ondas}

En la regi�n �ptica del espectro, un fotodetector responde directamente a la potencia �ptica que incide sobre su superficie. Por lo tanto, para un detector semiconductor, si la potencia �ptica $P$ incide sobre la regi�n fotosensible, la absorci�n de un fot�n genera un electr�n en la banda de conducci�n y un hueco en la banda de valencia. Bajo la influencia de campos internos y aplicados, el hueco y el electr�n se mueven en direcciones opuestas, lo que genera una fotocorriente $i$ que es la respuesta al fot�n absorbido incidente. En la mayor�a de las circunstancias, la fotocorriente es linealmente proporcional a la potencia incidente.

\begin{equation}
i = R\, P
\end{equation}

La constante de proporcionalidad $R$ se denomina capacidad de respuesta del detector y se expresa mediante

\begin{equation}
R=\frac{n_{qe} q}{h v}
\end{equation}

\subsection{El principio de Huygens-Fresnel en coordenadas rectangulares}

Antes de introducir una serie de aproximaciones al principio de Huygens-Fresnel, ser� �til enunciar primero el principio de forma m�s expl�cita para el caso de coordenadas rectangulares. Como se muestra en la figura 4.1, se supone que la apertura de difracci�n se encuentra en el plano $(\xi, \eta)$ y est� iluminada en la direcci�n z positiva. Calcularemos el campo de ondas a trav�s del plano $(x, y)$, que es paralelo al plano $(\xi, \eta)$ y a una distancia normal $z$ de �l. El eje $z$ atraviesa ambos planos en sus or�genes.

\end{document}